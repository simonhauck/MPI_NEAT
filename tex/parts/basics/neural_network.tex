% !TeX spellcheck = de_DE
\section{Neuronale Netze}
\ac{KNN} sind ein Teilgebiet des maschinellen Lernens, deren Geschichte 1943 mit \citeauthor{mcculloch1943logical} beginnt, welche in ihrer Arbeit ein einfaches neuronales Netz mit Schwellwerten entwickeln. Dieses ist von biologischen neuronalen Netzen inspiriert und ermöglicht die Berechnung von logischen und arithmetischen Funktionen \cite{mcculloch1943logical}. In den folgenden Jahrzehnten wird die Funktionsweise der neuronalen Netze weiterentwickelt und der Einsatz in verschiedensten Aufgabenfeldern ermöglicht. Hierzu zählen unter anderem die Klassifizierung von Bildern \cite{krizhevsky2012imagenet}, Erkennen und Interpretation von Sprache \cite{hinton2012deep}, \cite{andor2016globally} sowie das selbständiges Lösen von Computer- und Gesellschaftsspielen \cite{mnih2013playing}, \cite{silver2016mastering}. 

In diesem Kapitel wird zuerst ...
\subsection{Das Vorbild - Biologische neuronale Netze}
\subsection{Das Neuron}
\subsection{Netzstrukturen}
\subsection{Optimierungsverfahren}

Das Gebiet der Künstlichen Neuronalen Netze wird bereits seit 1943 erforscht