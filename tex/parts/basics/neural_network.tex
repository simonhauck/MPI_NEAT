% !TeX spellcheck = de_DE
\section{Neuronale Netze}
Klassische Algorithmen in der Informatik beschreiben mit welchen Schritten ein spezielles Problem gelöst werden kann. In vielen Anwendungsfällen, wie zum Beispiel beim Sortieren einer Liste, verwenden Computersysteme diese und lösen das gegebene Problem schneller und effizienter als es Menschen möglich ist. 

Dennoch gibt es Aufgaben, die von Menschen ohne Aufwand gelöst werden, aber Computersysteme vor große Herausforderungen stellen. Hierzu zählt unter anderem die Klassifizierung von Bildern. Ein Mensch kann Bilder von Hunden und Katzen unabhängig von Blickwinkel und Bildqualität unterscheiden beziehungsweise richtig zuordnen. Trotzdem lassen sich für solche Probleme keine klassischen Algorithmen finden, da die Lösung von vielen subtilen Faktoren abhängt \cite{kriesel2008kleiner}.

In vielen dieser Aufgabenfelder werden \ac{KNN} eingesetzt, welche von den biologischen neuronalen Netzen inspiriert sind und zum Forschungsgebiet des maschinellen Lernens gehören. Die Grundlage für die \ac{KNN} bildet die Arbeit von \citeauthor{mcculloch1943logical}, in der sie 1943 ein einfaches neuronales Netz mit Schwellwerten entwickelt haben. Dies ermöglicht die Berechnung von logischen und arithmetischen Funktionen \cite{mcculloch1943logical}. In den folgenden Jahrzehnten wird die Funktionsweise der neuronalen Netze weiterentwickelt und der Einsatz in verschiedensten Aufgabenfeldern ermöglicht. Hierzu zählen neben der Klassifizierung von Bildern \cite{krizhevsky2012imagenet} unter anderem das Erkennen und die Interpretation von Sprache \cite{hinton2012deep}, \cite{andor2016globally} sowie das selbständiges Lösen von Computer- und Gesellschaftsspielen \cite{mnih2013playing}, \cite{silver2016mastering}. 

In diesem Kapitel wird zuerst ...
\subsection{Biologische neuronale Netze}
Wie bereits beschrieben orientiert sich das Fachgebiet der \ac{KNN} an den erfolgreichen biologischen neuronalen Netzen, wie zum Beispiel dem menschlichen Gehirn \cite{kriesel2008kleiner}. In diesem Abschnitt werden die Eigenschaften betrachtet, die das Vorbild erfolgreich machen und für die \ac{KNN} übernommen werden sollen. Im Zuge dessen wird ein grober Überblick über die Struktur und Funktionsweise des menschlichen Gehirns gegeben. 
\\\\
Jede Sekunde erfassen die Rezeptoren des menschlichen Körpers unzählige Reize, wie zum Beispiel Licht, Druck, Temperatur und Töne. Die Reize werden anschließend elektrisch oder chemisch kodiert und über Nervenbahnen an das Gehirn geleitet, welches die Aufgabe hat diese zu filtern, zu verarbeiten und entsprechend zu reagieren. Als Reaktion können zum Beispiel Signale an entsprechende Muskeln oder Drüsen gesendet werden \cite{kinnebrock2018neuronale}. 

Bevor im nächsten Kapitel die Funktionsweise des Gehirns näher betrachtet wird, sollen drei Eigenschaften genannt werden, die klassische Algorithmen nicht besitzen beziehungsweise nur schwer umsetzen können, aber für biologische neuronale Netze keine Herausforderung sind. Ziel ist es, diese für die \ac{KNN} zu übernehmen \cite{kriesel2008kleiner}.
\begin{enumerate}
	\item \textbf{ Fähigkeit zu Lernen:} \\
	Das menschliche Gehirn ist nicht wie ein klassischer Algorithmus für seine Aufgaben programmiert. Stattdessen besitzt es die Fähigkeit anhand von gegebenen Beispielen und oder einfachem Ausprobieren zu lernen \cite{kriesel2008kleiner}. Hierbei wird das gewünschte Ergebnis mit dem tatsächlich erhaltenen Ergebnis verglichen und das Verhalten entsprechend angepasst. Dies ermöglicht es Menschen verschiedenste Aufgabengebiete erfolgreich zu lösen und sich ändernden Anforderungen anzupassen.
	
	\item \textbf{Fähigkeit zur Generalisierung:}\\
	Allerdings kann nicht jedes mögliche Szenario für ein Aufgabenfeld durch Ausprobieren oder Beobachtung gelernt werden. Trotzdem trifft das Gehirn in den meisten Situationen plausible Lösungen, da es die Fähigkeit zur Generalisierung besitzt 
	\cite{kriesel2008kleiner}. Das bedeutet, dass viele Situationen bereits bekannten Problemen zugeordnet werden können, mithilfe derer eine passende Verhaltensstrategie ausgewählt wird. 
	
	\item \textbf{Toleranz gegenüber Fehlern}\\
	Die Fähigkeit zu Generalisieren erlaubt auch eine hohe Fehlertoleranz gegenüber verrauschten Daten. Bei dem oben genannten Beispiel der Klassifizierung von Bildern kann ein Teil des Bildes fehlen oder unscharf sein und trotzdem kann das abgebildete Motiv richtig zugeordnet werden.
\end{enumerate}

\subsubsection{Struktur des menschlichen Gehirns}
Das menschliche Gehirn kann die genannten Eigenschaften durch seine besondere Struktur und Informationsverarbeitung erzielen. Diese wird im Folgenden oberflächlich erläutert, da dies die Grundlage für der \ac{KNN} ist. Für ein tieferes biologisches Verständnis wird auf entsprechende Fachliteratur verwiesen.
% TODO Fachliteratur verlinken
Das menschliche Gehirn kann in mehrere Regionen unterteilt werden. Das Großhirn
Das menschliche Gehirn besteht aus ${10}^{11}$ einzelnen Neuronen, welche unabhängig voneinander operieren. Jedes Neuron kann zu 

\subsection{Das Neuron}
\subsection{Netzstrukturen}
\subsection{Optimierungsverfahren}

Das Gebiet der Künstlichen Neuronalen Netze wird bereits seit 1943 erforscht