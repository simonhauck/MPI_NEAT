% !TeX spellcheck = de_DE
\section{Testumgebung}
\label{sec:analysis_testsetup}
Bevor die verschiedenen Optimierungsprobleme getestet werden, wird an dieser Stelle zunächst auf die Testumgebung eingegangen. Hierfür sind verschiedene Anforderungen zu definieren. Da in dieser Arbeit die Ausführung auf einem verteilten System betrachtet werden soll, müssen grundsätzlich mehrere Geräte zur Verfügung stehen, welche über ein Netzwerk miteinander verbunden sind. Ein aussagekräftiger Vergleich zwischen der parallelisierten und sequenziellen Implementierung ist einfacher, wenn alle Geräte des Clusters dieselbe Hardware verwenden. Bieten diese dieselbe Leistung, steht dem Cluster mit der doppelten Anzahl an Geräten auch die doppelte Rechenkapazität zur Verfügung. Als letzte Anforderung muss das System sowohl in der Anschaffung als auch im Betrieb kostengünstig sein.
\\\\
Ein Ansatz zur Umsetzung einer solchen Testumgebung ist die Nutzung von mehreren virtuellen Servern. Für diese gibt es zahlreiche kostengünstige Angebote von verschiedenen Anbietern. Allerdings kann bei solchen Systemen nicht auf die zugrunde liegende Hardware Einfluss genommen werden, wodurch ein Vergleich der verschiedenen Implementierungen schwierig sein kann. Aus diesem Grund ist dieser Ansatz für den beschriebenen Anwendungskontext ungeeignet und wird nicht weiter verfolgt. Eine Alternative ist die Anschaffung mehrerer physischer Geräte, welche dieselben Komponenten verwenden. Wie in Kapitel \ref{subsubsec:beowulf_cluster} beschrieben, kann mit diesen ein Beowulf Cluster erstellt werden, der das kostengünstige Testen von Anwendungen im Bereich \ac{HPC} ermöglicht. Da in dieser Arbeit primär der Vergleich zwischen der sequenziellen und parallelisierten Implementierung untersucht wird, ist die absolute Leistung nicht das entscheidende Kriterium. Der Fokus liegt auf den Anschaffungs- und Betriebskosten sowie der platzsparenden Unterbringung. Hierfür bieten sich Raspberry Pis besonders gut an.  
\\\\
In diesem Projekt wird als Basis das Modell 4 mit insgesamt 4GB \ac{RAM} verwendet, auf welchen standardmäßig ein 64bit Quad-core ARM Prozessor verbaut ist \cite{raspberryspecs}. Zusätzlich kann der später erstellte Cluster von der Gigabit Ethernet Verbindung profitieren, welche in diesem Modell neu hinzugefügt wurde und eine ausreichend große Bandbreite für eine effiziente Kommunikation bietet. Die Gesamtkosten für einen Raspberry Pi mit dieser Konfiguration belaufen sich zum Zeitpunkt der Arbeit auf ca. 60 Euro. Als Betriebssystem wird Raspian (Version 10) verwendet. Zwar ist dieses Standardbetriebssystem nur in einer 32bit Variante verfügbar und dadurch in vielen Fällen langsamer als andere 64bit Betriebssysteme, dennoch überwiegen die Vorteile für diese Arbeit. Raspian ist weit verbreitet und bietet eine hohe Kompatibilität zu diversen Softwarepaketen, wie zum Beispiel Tensorflow, von welchen zukünftige Erweiterungen profitieren können. Für die eigentliche Ausführung des implementierten Verfahrens wird Python in der Version 3.7.3 verwendet. Die erforderlichen Bibliotheken sind in der Datei \emph{requirements.txt} im Projekt spezifiziert und in einer virtuellen Umgebung installiert. Mit diesem Aufbau wird die Analyse des sequenziellen Verfahrens durchgeführt.
