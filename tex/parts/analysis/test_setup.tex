% !TeX spellcheck = de_DE
\section{Testumgebung}
\label{sec:analysis_testsetup}
Bevor die verschiedenen Optimierungsprobleme getestet werden können, wird in diesem Abschnitt kurz auf die Testumgebung eingegangen. Hierfür sind zunächst einige Anforderungen zu definieren. Da in dieser Arbeit die Ausführung auf einem verteilten System betrachtet werden soll, müssen grundsätzlich mehrere Geräte zur Verfügung stehen, welche über ein Netzwerk miteinander verbunden sind. Zusätzlich ist ein aussagekräftiger Vergleich zwischen der parallelisierten und sequenziellen Implementierung einfacher, wenn alle Geräte des Clusters dieselbe Hardware verwenden. Bieten diese dieselbe Leistung gilt für den Cluster, dass mit der doppelten Anzahl an Geräten auch die doppelte Rechenkapazität zur Verfügung steht. Die letzte Anforderung ist, dass das System kostengünstig in der Anschaffung und im Betrieb sein muss. 
\\\\
Ein Ansatz um eine solche Testumgebung umzusetzen besteht in dem Nutzen von mehreren virtuellen Servern, für welche es zahlreiche kostengünstige Angebote von verschiedenen Anbietern gibt. Allerdings kann bei solchen Systemen nicht auf die zugrunde liegende Hardware Einfluss genommen werden, wodurch ein Vergleich zwischen den verschiedenen Implementierungen schwierig sein kann. Aus diesem Grund ist dieser Ansatz für den beschriebenen Anwendungskontext ungeeignet und wird nicht weiter verfolgt. Eine Alternative ist das Anschaffen von mehreren physischen Geräten, welche dieselben Komponenten verwenden. Wie in Kapitel \ref{subsubsec:beowulf_cluster} beschrieben, kann mit diesen ein Beowulf Cluster erstellt werden, der das kostengünstige Testen von Anwendungen im Bereich \ac{HPC} ermöglicht. Da in dieser Arbeit primär der Vergleich zwischen der sequenziellen und parallelisierten Implementierung untersucht wird, ist die absolute Leistung nicht das entscheidende Kriterium, stattdessen wird auf die Anschaffungs- und Betriebskosten sowie eine platzsparende Unterbringung geachtet. Hierfür bieten sich Raspberry Pis besonders gut an.  
\\\\
In diesem Projekt wird als Basis das Modell 4 mit insgesamt 4GB \ac{RAM} verwendet, auf welchen standardmäßig ein 64bit Quad-core ARM Prozessor verbaut ist \cite{raspberryspecs}. Zusätzlich kann der spätere erstellte Cluster von der Gigabit Ethernet Verbindung profitieren, welche in diesem Modell neu hinzugefügt wurde und eine ausreichend große Bandbreite eine effizientere Kommunikation bietet. Die Gesamtkosten für einen Raspberry Pi mit dieser Konfiguration belaufen sich zum Zeitpunkt der Arbeit auf knapp 60 Euro. Als Betriebssystem wird Raspian (Version 10) verwendet. Zwar ist dieses Standardbetriebssystem nur in einer 32bit Variante verfügbar und demnach in vielen Fällen langsamer als andere 64bit Betriebssysteme, dennoch überwiegen die Vorteile für diese Arbeit. Raspian ist weit verbreitet und bietet eine hohe Kompatibilität zu diversen Softwarepaketen, wie zum Beispiel Tensorflow, von welchen zukünftige Erweiterungen profitieren können. Für die eigentliche Ausführung des implementierten Verfahrens wird Python in der Version 3.7.3 verwendet. Die erforderlichen Bibliotheken sind in der Datei \emph{requirements.txt} im Projekt spezifiziert und in einer virtuellen Umgebung installiert. Mit diesem Aufbau wird die Analyse des sequenziellen Verfahrens durchgeführt.
